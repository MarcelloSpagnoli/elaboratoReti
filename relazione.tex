\documentclass[a4paper,11pt]{article}

% --- Encoding & lingua ---
\usepackage[T1]{fontenc}
\usepackage[utf8]{inputenc}
\usepackage[italian]{babel}

% --- Font moderno con supporto completo ---
\usepackage{lmodern}

% --- Layout & stile ---
\usepackage{geometry}
\geometry{margin=2.5cm}
\usepackage{titlesec}
\titleformat{\section}{\normalfont\Large\bfseries}{\thesection}{1em}{}

% --- Colori, link e codice ---
\usepackage{xcolor}
\usepackage{hyperref}
\hypersetup{colorlinks=true, urlcolor=blue, linkcolor=black}

% --- Per codice sorgente ---
\usepackage{listings}
\lstset{
  basicstyle=\ttfamily\small,
  breaklines=true,
  frame=single,
  columns=fullflexible
}

% --- Tabelle ---
\usepackage{array}
\usepackage{booktabs}

% --- Titolo ---
\title{Relazione – Web Server in Python}
\author{\textbf{Marcello Spagnoli}\\
Matricola: 0001117244\\
\texttt{marcello.spagnoli2@studio.unibo.it}}

\begin{document}

\maketitle

\section*{1. Richieste della consegna}
Progettare un semplice server HTTP in Python (usando \texttt{socket}) e servire un sito web statico con HTML/CSS.

\section*{Requisiti minimi}
\begin{itemize}
  \item Il server deve rispondere su \texttt{localhost:8080}.
  \item Deve servire almeno 3 pagine HTML statiche.
  \item Gestione di richieste \texttt{GET} e risposta con codice 200.
  \item Implementare risposta 404 per file inesistenti.
\end{itemize}

\section*{Estensioni opzionali}
\begin{itemize}
  \item Gestione dei MIME types (\texttt{.html}, \texttt{.css}, \texttt{.jpg}, ecc.).
  \item Logging delle richieste.
  \item Aggiunta di animazioni o layout responsive.
\end{itemize}

\section*{Output atteso}
\begin{itemize}
  \item Codice del server Python.
  \item Cartella \texttt{www/} con i file HTML/CSS.
  \item Relazione tecnica.
\end{itemize}


\section*{Funzionalità implementate}
\begin{itemize}
  \item Gestione delle richieste \texttt{GET} con risposta 200.
  \item Gestione file non trovati con risposta 404.
  \item Rifiuto dei metodi diversi da GET con errore 405.
  \item Gestione di richieste malformate con errore 400.
  \item Gestione eccezioni interne con errore 500.
  \item Logging su terminale e su file con timestamp.
  \item Riconoscimento automatico del MIME type.
  \item Animazioni e layout responsive
\end{itemize}

\section*{Contenuti statici}
Nel progetto sono presenti 3 pagine HTML accessibili:
\begin{itemize}
  \item \texttt{index.html}
  \item \texttt{about.html}
  \item \texttt{contact.html}
\end{itemize}
Tutti i file si trovano nella cartella \texttt{www/}. È presente anche il template \texttt{error.html}, utilizzato per mostrare i messaggi di errore personalizzati.
\\È presenta una cartella css dove si trova un general.css che viene applicato a tutte le pagine + 3 file css specifici per ogni pagina

\section*{Funzionamento}

Il server HTTP è stato implementato in Python utilizzando il modulo \texttt{socket} per la comunicazione e \texttt{threading} per la gestione concorrente dei client. Di seguito si riassume il flusso di funzionamento:

\begin{itemize}
  \item All'avvio, il server effettua il \textbf{bind} sulla porta 8080 e resta in ascolto su \texttt{localhost}.
  \item Ogni volta che un client si connette, viene avviato un nuovo thread che gestisce la connessione tramite la funzione \texttt{handle\_client}.
  \item Il server legge la prima riga della richiesta HTTP e ne estrae metodo, path e protocollo.
  \item Se la richiesta non è ben formata (ad esempio mancano elementi), il server restituisce un errore \textbf{400 Bad Request}.
  \item Se il metodo non è \texttt{GET}, viene restituito un errore \textbf{405 Method Not Allowed}.
  \item Se il file richiesto esiste all'interno della cartella \texttt{www}, viene restituito con codice \textbf{200 OK} e relativo MIME type.
  \item Se il file non esiste, il server restituisce un errore \textbf{404 Not Found}, generando dinamicamente la pagina tramite un template HTML.
  \item Se durante il processo si verifica un errore imprevisto, viene inviato un messaggio \textbf{500 Internal Server Error}.
\end{itemize}

Ogni richiesta viene loggata sia sul terminale che in un file \texttt{server.log}, indicando timestamp, tipo di richiesta e codice di risposta. Le pagine di errore vengono generate dinamicamente usando il file \texttt{error.html} come template, sostituendo il messaggio nella variabile \texttt{\{\{ message \}\}}.
\\Tramite CSS sono state incluse animazioni e è stato implementato un layout responsive adatto al ridimensionamento


\section*{Test effettuati}
Ho effettutato un test scrivendo un file \texttt{test\_errori.py} che prova tutti i seguenti errori mandando richieste errate o non supportate per qualche motivo:\\\\
\begin{tabular}{@{}lll@{}}
\toprule
\textbf{Test} & \textbf{Metodo} & \textbf{Risultato Atteso} \\
\midrule
File esistente & GET \texttt{/index.html} & 200 OK con contenuto \\
File inesistente & GET \texttt{/nofile.html} & 404 Not Found \\
Metodo non supportato & POST \texttt{/index.html} & 405 Method Not Allowed \\
Richiesta malformata & GET (e basta) & 400 Bad Request \\
Errore interno & GET /test500.html (non ha permessi di lettura) & 500 Internal Server Error \\
\bottomrule
\end{tabular}

\section*{Estensioni opzionali implementate}
\begin{itemize}
  \item Gestione dinamica del tipo MIME con il modulo \texttt{mimetypes}.
  \item Logging dettagliato (IP client, metodo, percorso, stato).
  \item Template per errori con messaggio dinamico (\texttt{\{\{ message \}\}}).
  \item Utilizzo di \texttt{threading.Thread} per supportare più client contemporaneamente.
\end{itemize}


\section*{Avvio del Server e Requisiti}

Per avviare il server, è necessario avere Python 3 installato.  
Il server si avvia eseguendo il file \texttt{server.py} da terminale con il comando:

\begin{verbatim}
python3 server.py
\end{verbatim}

Il server ascolterà sulla porta 8080 e servirà i file contenuti nella cartella \texttt{www/}.

\section*{Istruzioni per l'avvio e il test del server}
\begin{enumerate}
  \item Assicurarsi di avere installato Python 3.6 o versione superiore.
  \item Posizionarsi nella cartella \texttt{elaboratoReti} dove si trova il file \texttt{server.py} e la cartella \texttt{www}.
  \item Avviare il server eseguendo:
  \begin{quote}
    \texttt{python3 server.py}
  \end{quote}

  \item Una volta avviato, il server sarà in ascolto sulla porta 8080. Si può testare il funzionamento aprendo un browser e visitando:
  \begin{quote}
    \texttt{http://localhost:8080/index.html}
  \end{quote}

  \item Per provare i vari casi di errore, si possono usare strumenti come Thunder Client, provare a richiedere pagine che non esistono dal browser o provare il file \texttt{test\_errori.py}.
  \item Tutte le richieste e gli errori verranno registrati nel file \texttt{server.log} nella stessa cartella di \texttt{server.py}.
  \item Per fermare il server premere \texttt{Ctrl+C} nel terminale dove è in esecuzione.
\end{enumerate}



\subsection*{Struttura delle cartelle}

\begin{verbatim}
elaboratoReti/
|-- server.py
|-- server.log
|-- www/
    |-- index.html
    |-- error.html
    |-- contact.html
    |-- test500.html (file vuoto solo per il test dell'errore 500, togliere i permessi)
    |-- gear.html
    |-- css/
    |-- img/
\end{verbatim}

Dove:
\begin{itemize}
  \item \texttt{server.py} è il file principale che avvia il server
  \item \texttt{server.log} contiene i log delle richieste e degli errori
  \item \texttt{www/} contiene le pagine web statiche servite dal server
\end{itemize}


\end{document}
